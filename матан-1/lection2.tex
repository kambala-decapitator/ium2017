\section{Отношения порядка, аксиома выбора, теорема Цермело, отношение эквивалентности, поле $\Q$, аксиома полноты. (11.09.2017)}

Как ввести отношение порядка на $\N$? $n \le m \Leftrightarrow n=m \lor m=n+k$. Но операции сложения в общем виде у нас еще нет, есть только увеличение на $1$.

Докажем, что $\forall n\, \exists! f_n(m): \N \to \N$ со свойствами: $f_n(1) = n+1$ и $f_n(m+1) = f_n(m) + 1$, тогда сложение можно определить как $n+m = f_n(m)$. Доказательство следует из того факта, что функция задана рекуррентным соотношением. (также $f_{n+1}(m) = f_n(m) + 1$)\\

Пусть есть некое рекуррентное правило: $\forall n\, F_n: A^n \to A$ и $a_1$~-- задано, а $a_{n+1} = F_n(a_1, \dots, a_n)$. Это частый способ задания последовательностей. (например, числа Фибоначчи)
\begin{claim}
	Существует единственная такая последовательность $a_1, \dots, a_n$.
\end{claim}
\begin{proof}
	$a_1$~-- задано, $a_2 = F_1(a_1), a_3 = F_2(a_1, a_2), \dots$ По сути надо доказать, что $\forall m\, \exists! \tilde{a}_1, \dots, \tilde{a}_m$. Докажем по индукции. Для $m=1$: $\tilde{a}_1 = a_1$; пусть доказано для $m$, проверим для $m+1$: $\tilde{a}_1, \dots, \tilde{a}_m \stackrel{F}{\longrightarrow} a_{m+1}$, где $\tilde{a}_1, \dots, \tilde{a}_m$ определены однозначно. Осталось построить исходную последовательность: $a_m = \tilde{a}_m$.
\end{proof}

Обобщим это утверждение на случай вполне упорядоченных множеств.
\begin{theo}
	Пусть $A$~-- вполне упорядоченное множество, $\{a: a < c\}$~-- начальный полуотрезок $[0,c)\, \forall c$ и задано рекурсивное правило $F$, которое на вход получает $c$ и $g|_{[0,c)} \to B$ ($B$~-- произвольное непустое множество) и выдает $b \in B$. Тогда $\exists! f: A \to B: f(c) = F(c; f|_{[0,c)})$.
\end{theo}
\begin{proof}
	Аналогично предыдущему утверждению.
\end{proof}

\begin{cor}
	Если $A, B$~-- вполне упорядоченные множества, то $A \sim B' \subseteq B \lor B \sim A' \subseteq A$.
\end{cor}
\begin{proof}
	Пусть $f: \{a: a<c\} \to B$, тогда $f(c) = \min{B \setminus f\big( [0,c) \big)}$ (получили рекурсивное правило), а если $B \setminus f\big( [0,c) \big) = \varnothing$, то сопоставляем $*$ (что угодно). По предыдущей теореме $\exists f: A \to B \cup \{*\}$, и есть два случая:
	\begin{enumerate}
		\item $*$ нет, тогда $f$ определена на всем $A$ (отображает $f$ в некоторое подмножество $B$);
		\item $*$ есть, тогда $\exists c: f(c) = *$. Возьмем среди всех $c$ минимальный и обозначим его $c_*$, тогда $f: [0, c_*) \stackrel{на}{\longrightarrow} B$.
	\end{enumerate}
	Убедимся, что $f$~-- биекция:
	\begin{itemize}
		\item проверим, что $f$ строго возрастает: возьмем $a_1 < a_2$ и проверим $f(a_2) \overset{?}{<} f(a_1)$. Это не так, ибо противоречит построению ($f(a_1)$ выбран не минимальный) $\Rightarrow f$~-- инъекция;
		\item проверим сюръективность: в случае 1 $f$ отображает $A$ в $B' \subseteq B$, а в случае 2 начальный полуотрезок $A$ (т.е. $A' \subseteq A$) отображается на $B$.
	\end{itemize}
	И утверждение следствия доказано.\\
	Теперь только проверим чему равно $f(A)$ в случае 1. Если $f(A)=B$, то все, иначе в $B \setminus f(A) \neq \varnothing$ возьмем минимальный $b$ и проверим, что $f(A) = [0,b)$: $f$~-- возрастающая функция и возьмем $c \in B \setminus f(A): c < b \Rightarrow$ противоречие построению.\pagebreak
\end{proof}

\textbf{Аксиома выбора}.
\begin{itemize}
	\item Пусть есть набор множеств $\{X_\alpha\}: X_\alpha \cap X_\beta = \varnothing, \alpha \neq \beta$, тогда $\exists C \subset \underset{\alpha}{\cup} X_\alpha: \forall \alpha\, C \cap X_\alpha$ состоит ровно из одного элемента.
	\item (эквивалентная формулировка) Пусть есть набор множеств $\{X_\alpha\}: X_\alpha \neq \varnothing$, тогда $\exists f: \{X_\alpha\} \to \underset{\alpha}{\cup} X_\alpha: f(X_\alpha) \in X_\alpha$. (переход из предыдущей формулировки в эту: $X_\alpha \mapsto \tilde{X}_\alpha = \{(x,\alpha) | x \in X_\alpha\}$)
\end{itemize}

\begin{theo}[Цермело]
	Всякое множество можно вполне упорядочить.
\end{theo}
\begin{proof}
	Пусть есть множество $A$ и зададим функцию $f: 2^A \setminus A \to A: f(B) \in A \setminus B, B \neq A$. Зададим порядок: $a_0 = f(\varnothing), a_1 = f(\{a_0\})$ «и так далее». Теперь из таких цепочек надо «сшить» порядок: возьмем вполне упорядоченное множество $B \subset 2^A \setminus A: f\big( [0,b) \big) = b$. Теперь если взять два таких вполне упорядоченных множества, обязательно окажется, что одно из них является начальным полуотрезком другого. И когда все такие множества объединить, получится вполне упорядоченное. (см. Верещагин, Шень)
\end{proof}
\begin{cor}
	$\forall A,B \quad A \sim B' \subseteq B \lor B \sim A' \subseteq A$.\\
\end{cor}

\begin{defin}
	$R \subset X \times X$ называется \underline{отношением эквивалентности}, если выполнены свойства:
	\begin{enumerate}
		\item рефлексивность: $(x,x) \in R$;
		\item симметричность: $(x,y) \in R \Rightarrow (y,x) \in R$;
		\item транзитивность: $(x,y) \in R \land (y,z) \in R \Rightarrow (x,z) \in R$.
	\end{enumerate}
\end{defin}
В дальнейшем вместо $(x,y) \in R$ будем писать $x \sim y$.

\begin{defin}
	$R(a) = \{x: x \sim a\}$ называется \underline{классом эквивалентности с представителем $a$}.
\end{defin}
\begin{claim}
	$R(a) \cap R(b) \Rightarrow R(a) = R(b)$.
\end{claim}
\begin{proof}
	Следует из свойства транзитивности отношения эквивалентности.
\end{proof}
\begin{defin}
	Объединением всех классов эквивалентности называется \underline{фактормножество} и обозначается $X/\!\sim$.
\end{defin}

\begin{defin}
	Рассмотрим $\Z \times \N$ с отношением эквивалентности: $(m,n) \sim (p,q) \Leftrightarrow mq=np$. Такое множество классов эквивалентности называется \underline{множеством рациональных чисел} и обозначается $\Q$.\\
\end{defin}

Множество вещественных чисел построим аксиоматически.
\begin{defin}
	Множество $F$ с операциями $+$ и $\cdot$ называется \underline{полем}, если:
	\begin{enumerate}
		\item $(F,+)$~-- абелева группа;
		\item $(F \setminus \{0\}, \cdot)$~-- абелева группа;
		\item $a(b+c) = (b+c)a = ab+ac$ (дистрибутивность).
	\end{enumerate}
\end{defin}
\begin{ex}
	\begin{itemize}
		\item $\Q$ (поле характеристики $0$);
		\item $\Z_p = \{0,1,\dots,p-1\}$~-- кольцо вычетов по $p$, $p$~-- простое (поле характеристики $p$).
	\end{itemize}
\end{ex}

\begin{defin}
	\underline{Упорядоченное поле}~-- это поле, на котором определен линейный порядок, причем он согласуется с операциями:
	\begin{enumerate}
		\item $c \ge 0, a \le b \Rightarrow ac \le bc$;
		\item $a \le b \Leftrightarrow a+c \le b+c$.\\
	\end{enumerate}
\end{defin}

Пусть есть упорядоченное поле. $1 > 0$?
\begin{proof}
	(от противного) $1 < 0$, отнимем $1$: $0 < -1$. Так как $-1 > 0$, домножим на $-1$ и получим $0 < 1$~-- противоречие.\\
\end{proof}

К сожалению, $\R$ не является упорядоченным полем (но $\Q$ является).

\begin{defin}
	$A$ \underline{левее} $B$, если $a \le b \quad \forall a \in A, b \in B$.
\end{defin}

\textbf{Аксиома полноты}. Если есть два непустых множества $A$ и $B$ такие, что $A$ левее $B$, то существует $c$, который их разделяет, то есть $A$ левее $\{c\}$ и $\{c\}$ левее $B$.
\begin{defin}
	Упорядоченное поле, на котором выполняется аксиома полноты, называется \underline{множеством вещественных (действительных) чисел} и обозначается $\R$.
\end{defin}